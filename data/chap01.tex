% !TeX root = ../main.tex

\chapter{绪论}

面对XXX带来的挑战,本文提出了一种XXX方法。
本章节针对热点新闻事件脉络分析的研究背景、意义、现状,以及本文的研究内容与研究思路进行了简要介绍。

\section{研究背景及意义}

\subsection{研究背景}

热点新闻事件一般指社会上近期发生的引起大众关注的重要事件,随着互联网和社交媒体的快速兴起,普通受众和新闻工作者很难快速、高质量地从海量的新闻报道中获取有效信息,厘清热点事件的因果和发展过程。
所以针对热点新闻的事件脉络分析任务应运而生,该任务旨在检测和追踪一段时间内新闻话题事件的发展和演变情况,便于大众尽快地了解整个新闻事件的全貌,同时有助于新闻工作者深层次掌握和监控新闻事件的发展脉络和趋势,从而对新闻事件的舆情进行监测和研判预警。

(如图~\ref{fig:newzoo-player-forecast})
\begin{figure}[htb]
  \centering
  % 这里可以控制图片宽度比例
  \includegraphics[width=0.8\linewidth]{cat.jpg}
  \caption{2015-2024年全球玩家数\cite{newzoo2021report}}
  \label{fig:newzoo-player-forecast}
\end{figure}

\subsection{研究意义}

% https://www.zhihu.com/question/393670234

伴随互联网的飞速发展,网络媒体以实时、快速、易传播的特点蓬勃发展,新闻事件的传播力度大大增加,人们可以即时获取全球各地的多元信息,实时掌握事件的最新动态。在众多的新闻报道中,热点事件是引起广泛讨论和关注的重大事件,产生的社会影响巨大。这类新闻事件存在爆发迅速、持续时间长、信息规模大、数据稀疏等特征,加上个性化内容的精确推荐,大众聚焦事件核心和全面把控信息的效率也随之降低。

而事件脉络分析任务的兴起就是为了揭示热点事件的发展和演变规律,满足大众和研究人员对深入了解和跟踪新闻话题事件发展的需求,同时为决策者提供有价值的参考和洞察。从社会需要的角度来看,事件脉络分析任务可以帮助公众全面认知社会热点事件,促进信息传播和知情权的实现。而从学术价值的角度来看,该任务为研究人员提供了研究新闻舆情和媒体报道影响力的机会,有助于推动新闻传播领域的理论发展和实证研究。
因此,事件脉络分析在满足公众信息需求的同时,也具有重要的学术研究意义。

\section{国内外研究现状}

为了对新闻报道中的事件信息进行有效、自动地提取,实现对事件脉络发展的分析,业界做出了很多相关研究。话题检测与追踪(Topic Detection and Tracking,TDT)被认为是最早可以自动识别和追踪新闻文本数据中的话题的任务,它起源于20世纪90年代初期,旨在从海量的新闻报道中自动提取出涉及某一具体事件的相关报道\cite{TDT}。到2010年代中期,TDT技术逐渐与情感分析、实体识别等自然语言处理技术融合,形成了更为综合化的分析框架。TDT 主要包含三个子任务:(1)报道切分:将数据流划分为不同的故事;(2)话题检测:从多个新闻报道中检测出属于同一个话题的新闻事件;(3)话题追踪:在故事已有的事件阶段的基础上,对与该话题相关的后续事件进行追踪。但是TDT存在对话题定义不清、算法复杂度高等问题,同时因为TDT难以用结构化的方式展示事件之间的演变关系,所以提取得到的新闻事件脉络缺乏一定的解释性\cite{TDT-disad}。因此,为了增强事件发展脉络的准确性、完整性、连贯性和可解释性,研究人员进一步提出了故事脉络生成(Storyline Generation,SG)任务\cite{SG}。故事脉络生成旨在将来自多个新闻源的新闻报道组织整合成一个完整、连续的话题故事。通过该任务,可以自动地分析和组织大量的新闻文本,提取出热点事件的全过程,并将其以一种逻辑和清晰的方式呈现给用户。事件脉络构建是事件脉络生成任务中最关键的环节,现有的事件脉络构建方法依据脉络的组建形式可主要分为时间线(Timeline)和故事线(Storyline)两大类,用户可以通过读取事件脉络,快速、直观地获得事件发展的全局信息。
基于时间线的脉络构建主要是按事件发展的时间顺序罗列主要事件。Ansah等人\cite{Ansah}提出了一种基于时间线摘要图(Story Graph)的脉络构建方法,对于一个特定的话题,首先通过命名实体识别和关系抽取来提取事件,然后使用时间线摘要图来表示事件发展过程,以节点表示事件,边表示它们之间的时间顺序。这种方法比传统的文本摘要方法更直观、易于理解,并且能够提供与时间相关的信息视图。而Huang等人\cite{CRP}提出了一个动态的中国餐厅过程(Chinese Restaurant Process, CRP)模型,该模型考虑了故事情节的出生、存活和死亡,更符合故事情节随时间发展的真实情况。
基于故事线的脉络按照构建角度的不同,可具体分为基于关联分析的事件脉络构建、基于特征建模的事件脉络构建、基于传播模型的事件脉络构建三类\cite{model-category}。

\subsection{论文主要部分的写法}

研究生学位论文撰写,除表达形式上需要符合一定的格式要求外,内容方面上也要遵循一些共性原则。

通常研究生学位论文只能有一个主题(不能是几块工作拼凑在一起),该主题应针对某学科领域中的一个具体问题展开深入、系统的研究,并得出有价值的研究结论。
学位论文的研究主题切忌过大,例如,“中国国有企业改制问题研究”这样的研究主题过大,因为“国企改制”涉及的问题范围太广,很难在一本研究生学位论文中完全研究透彻。

\subsection{论文的语言及表述}

除国际研究生外,学位论文一律须用汉语书写。
学位论文应当用规范汉字进行撰写,除古汉语研究中涉及的古文字和参考文献中引用的外文文献之外,均采用简体汉字撰写。

国际研究生一般应以中文或英文书写学位论文,格式要求同上。
论文须用中文封面。

研究生学位论文是学术作品,因此其表述要严谨简明,重点突出,专业常识应简写或不写,做到立论正确、数据可靠、说明透彻、推理严谨、文字凝练、层次分明,避免使用文学性质的或带感情色彩的非学术性语言。

论文中如出现一个非通用性的新名词、新术语或新概念,需随即解释清楚。

\subsection{论文题目的写法}

论文题目应简明扼要地反映论文工作的主要内容,力求精炼、准确,切忌笼统。
论文题目是对研究对象的准确、具体描述,一般要在一定程度上体现研究结论,因此,论文题目不仅应告诉读者这本论文研究了什么问题,更要告诉读者这个研究得出的结论。
例如:“在事实与虚构之间:梅乐、卡彭特、沃尔夫的新闻观”就比“三个美国作家的新闻观研究”更专业、更准确。

\section{本文研究内容}

总结本文研究内容。
本文的主要目的是……

一篇学位论文的引言大致包含如下几个部分:
1、问题的提出;
2、选题背 景及意义;
3、文献综述;
4、研究方法;
5、论文结构安排。
\begin{itemize}
  \item 问题的提出:要清晰地阐述所要研究的问题“是什么”。
  \item 选题背景及意义:论述清楚为什么选择这个题目来研究,即阐述该研究对学科发展的贡献、对国计民生的理论与现实意义等。
  \item 文献综述:对本研究主题范围内的文献进行详尽的综合述评,“述”的同时一定要有“评”,指出现有研究状态,仍存在哪些尚待解决的问题,讲出自己的研究有哪些探索性内容。
  \item 研究方法:讲清论文所使用的学术研究方法。
  \item 论文结构安排:介绍本论文的写作结构安排。
\end{itemize}

\section{本文组织结构}

本文通过以下六个章节详细论述该研究课题,结构内容如下:

第一章:绪论。本章讨论了XXX以及它所代表的XX方式的意义与可在经济、社会价值上起到的作用,
并回顾国内外现状,明确了本文研究内容与组织结构。

第二章:相关技术概述。本章对后续研究所要使用到的相关技术及概念进行了阐述。文中依次介绍了XXX、XXX、XX与XX等技术的基础原理与作用。...

第三章:XXX。...

第四章:XXX。...

第五章:系统设计与应用。本章从软件工程角度概述了系统的整体设计架构与设计方法,具体介绍了相关开发实现方案与特色功能,将三四章研究内容串联并应用以完成系统开发。
最后对系统进行了展示与测试评估。

第六章:总结与展望。本文目标内容已基本顺利完成,但其研究成果仍然存在许多改进空间。该部分对研究内容与系统成品进行回顾,总结了主旨与存在的问题。
通过思考相应改进措施,对未来研究内容与方向进行展望。
